% Use values that are applicable to you and your thesis.

\newif\ifenglish{}
\englishtrue{} % english language
% \englishfalse{} % slovak language

\newif\ifbachelor{}
\bachelortrue{} % bachelor's thesis
% \bachelorfalse{} % master's thesis

\newif\ifcompsci{}
\compscitrue{} % computer science
% \compscifalse{} % bioinformatics

\newif\ifconsultant{}
% \consultanttrue{} % I have a consultant
\consultantfalse{} % no consultant

\newif\iflogoFMFI{}
% \logoFMFItrue{} % display FMFI logo on the cover and the title page
\logoFMFIfalse{} % no logo

% The year in which you plan to turn the thesis in.
\newcommand{\thesisyear}{2022}

% Tip - use the `\\` to split a long thesis name nicely.
\newcommand{\thesisname}{Extraordinarily Long Example Name \\ of My Awesome Project}

% If you are writing a master's thesis, don't forget to add Bc. here.
\newcommand{\thesisauthor}{Adam~Fiktívny}

\newcommand{\thesissupervisor}{doc.~RNDr.~Boris~Fiktívny,~PhD.}

\ifconsultant{}
    \newcommand{\thesisconsultant}{doc.~RNDr.~Cyril~Fiktívny,~PhD.}
\fi

\newcommand{\thesisacknowledgments}{
    You can thank anyone who helped you with the thesis here (e.g.\ your supervisor).
}

\newcommand{\thesisabstractsk}{
    Slovenský abstrakt v rozsahu 100--500 slov, jeden odstavec. Abstrakt stručne sumarizuje výsledky práce. Mal by byť pochopiteľný pre bežného informatika. Nemal by teda využívať skratky, termíny alebo označenie zavedené v práci, okrem tých, ktoré sú všeobecne známe.
}

\newcommand{\thesisabstracten}{
    Abstract in the English language (translation of the abstract in the Slovak language).
}

\newcommand{\thesiskeywordssk}{Slovak, keywords, here}

\newcommand{\thesiskeywordsen}{English, keywords, here}

% https://tex.stackexchange.com/questions/7653/how-to-iterate-through-the-name-of-files-in-a-folder
\makeatletter
\def\app@exe{\immediate\write18}
\def\inputAllFiles#1#2{%
  \app@exe{ls #1/*.tex | xargs cat >> out/#2-inputs.tmp}%
  \InputIfFileExists{out/#2-inputs.tmp}{}{}
  \AtEndDocument{\app@exe{/bin/rm -f out/#2-inputs.tmp}}}
\makeatother

\newcommand{\thesischapters}{
    \inputAllFiles{./chapters}{chapters}
    % This inputs all files from the chapters directory, in alphabetical order.
    % You can comment it out and input the files manually if you prefer:
    % \input chapters/00-introduction.tex
    % ...
}

\newcommand{\thesisappendices}{
    \inputAllFiles{./appendices}{appendices}
}

% Thesis type, field, programme and department are set here automatically, but please check that
% the values are indeed correct for you.
% If your supervisor works at FMFI, set the \thesisdepartment to the supervisor's department (this
% one should also be present in the thesis assignment in AIS). Otherwise, keep the Department of
% Computer Science.
\ifenglish{}
    \ifbachelor{}
        \newcommand{\thesistype}{Bachelor's Thesis}
    \else
        \newcommand{\thesistype}{Master's Thesis}
    \fi
    \ifcompsci{}
        \newcommand{\thesisfield}{Computer Science}
        \newcommand{\thesisprogramme}{Computer Science}
    \else
        \newcommand{\thesisfield}{Computer Science and Biology}
        \newcommand{\thesisprogramme}{Bioinformatics}
    \fi
    \newcommand{\thesisdepartment}{Department of Computer Science}
    \newcommand{\thesisfaculty}{Faculty of Mathematics, Physics and Informatics}
    \newcommand{\thesisuniversity}{Comenius University in Bratislava}
\else
    \ifbachelor{}
        \newcommand{\thesistype}{Bakalárska práca}
    \else
        \newcommand{\thesistype}{Diplomová práca}
    \fi
    \ifcompsci{}
        \newcommand{\thesisfield}{Informatika}
        \newcommand{\thesisprogramme}{Informatika}
    \else
        \newcommand{\thesisfield}{Informatika a Biológia}
        \newcommand{\thesisprogramme}{Bioinformatika}
    \fi
    \newcommand{\thesisdepartment}{Katedra informatiky}
    \newcommand{\thesisfaculty}{Fakulta matematiky, fyziky a informatiky}
    \newcommand{\thesisuniversity}{Univerzita Komenského v Bratislave}
\fi
\newcommand{\thesislocation}{Bratislava}

% Helpers

\newif\ifshowframe{}
% \showframetrue{} % show page frames to debug positioning
\showframefalse{} % normal
