\chapter{Non-switching-isomorphic graphs}

If we hope to produce clean and usable data, we need to filter the graphs for switching-isomorphisms. Bagheri, Moghaddamfar, Ramezani\cite{switching-isomorphic} establish a method of determining the number of non-switching-isomorphic signed graphs based on the action of its automorphism group. We will now dive into the concepts of switching equivalence and automorphism on signed graphs and describe the algorithm used to generate non-switching-isomorphic graphs.

\section{Switching equivalence}

Given a graph $G$ there are $2^{|E_G|}$ possible signed graphs constructed from $G$. However, provided that $G$ is connected, only $2^{|E_G| - |V_G| + 1}$ or them are mutually non-switching equivalent.

\begin{theorem}\label{lem1:eq-classes}
    Let $G$ be a simple unsigned connected graph with $n$ vertices and $m$ edges. There are $2^{m - n + 1}$ mutually non-switching equivalent graphs on $G$.
\end{theorem}

\textit{Proof.} Bagheri, Moghaddamfar, Ramezani\cite{switching-isomorphic} also prove this theorem but we present a simpler version. The idea is to use a spanning tree $S \subseteq G$ and show that each switching equivalence class of $G$ has exactly one element that is all-positive on $S$. Since $S$ contains $n - 1$ edges, there are $2^{m - (n - 1)}$ different graphs all-positive on $S$. Suppose we have a signed graph $\Gamma$ all-positive on $S$ and we switch some vertices. If we switch no vertices or all vertices, the graph stays the same, so we will have a non-empty set of switched vertices $A$ and a non-empty set of unswitched vertices $B$. At least one edge of $S$ must have one end in $A$ and the other end in $B$, otherwise $G$ would not be connected or $S$ would not be a spanning tree. After this switching all edges with both ends in either $A$ or $B$ will retain the same sign (not reversed or reversed twice) and edges with one end in $A$ and on end in $B$ will have its sign reversed. Therefore every possible switching from $\Gamma$ will result in a graph that is not all-positive on $S$. \qed



\subsection{Automorphism}



A second approach is based on the cycle space of $\Gamma$ and Eulerian graphs.

\begin{theorem}[Zaslavsky, Cameron\cite{enumerating-switching-isomorphisms}]
    There is a one-to-one correspondence between switching isomorphism classes of $\Gamma$ and Aut($G$)-isomorphism classes of Eulerian subgraphs of $G$.
\end{theorem}

This approach results in a faster enumeration algorithm, but it does not provide means to generate non-isomorphic signed graphs. This is a point of possible future research as there might be a way to utilize this theory in a faster generation algorithm.

In this thesis we are using a simple approach for filtering signed graphs for isomorphisms. The signed graphs are converted to unsigned graphs differentiating negative edges by inserting a vertex in the middle. Now we can use known filtering algorithms, finding the canonical form for each new graph and comparing it against already seen graphs.

\begin{definition}
    A \textit{canonical form} is a labeled graph Canon($G$) that is isomorphic to $G$ such that every graph isomorphic to $G$ has the same canonical form. To compute whether graphs $G$ and $H$ are isomorphic we compute their canonical forms and test whether they are identical.
\end{definition}

Finding the canonical form is as hard as determining whether two graphs are isomorphic and as of today it is unknown if a polynomial deterministic algorithm exists that solves this problem. However, here we are working with small graphs (less than 26 vertices) and special cases of bigger graphs, so time costs related to the number of vertices are more or less irrelevant. What is relevant is that this approach has quadratic complexity with regards to the number of signed graphs on each base graph as we are comparing each new canonical form to potentially many previous ones. This cost is mitigated somewhat by filtering the graphs for signed snarks first and for isomorphisms second.
