\chapter*{Introduction}
\addcontentsline{toc}{chapter}{Introduction}
\markboth{Introduction}{Introduction}

The problem of graph colouring has been known for a long time and is still relentlessly being studied today. Even in a problem this wide there are still areas to explore and improve. Edge colouring in combination with the concept of signed graphs remains more or less unexplored.

First discovered by the mathematician Frank Harary in 1953 as a model for studying social networks, signed graphs remained idle until 1982 when Thomas Zaslavsky published multiple seminary papers on the topic. Many fundamental results in the study of nowhere-zero flows and the chromatic number of signed graphs have been established only recently and research the problem of edge colouring was started by Richard Behr in 2020. The goal of this thesis is to provide means of systematic research of 3-edge-colourability of signed cubic graphs by creating a generator for non-switching-isomorphic signed graphs and a database of small signed snarks.

In the first chapter we define key concepts in the signed graph theory and describe the current state of research. We also mention the relationship to unsigned graphs and how it affects the colour set and its requirements. In the second chapter we describe the programs that generate non-switching-isomorphic signed graphs and signed snarks. In the third and final chapter we present some results achieved by using these tools and suggest options for future research that can be pursued.
