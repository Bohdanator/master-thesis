
% \begin{figure}[h]
%     \centering
%     \begin{tikzpicture}
%         \begin{scope}[every node/.style={circle,thick,draw}]
%             \node (1) [fill=pink] at (0,0) {1};
%             \node (2) at (xyz polar cs:angle=0, radius=2.5) {2};
%             \node (3) at (xyz polar cs:angle=120, radius=2.5) {3};
%             \node (4) at (xyz polar cs:angle=240, radius=2.5) {4};
%         \end{scope}
%         \begin{scope}[every node/.style={colour=black}]
%             \node (1c) [above=1pt of 1, colour=red] {1};
%             \node (2c) [above=1pt of 2] {0};
%             \node (3c) [above=1pt of 3] {1};
%             \node (4c) [above=1pt of 4] {-1};
%         \end{scope}
%         \begin{scope}[every edge/.style={draw,very thick}]
%             \path
%                 (1) edge (2)
%                 (1) edge [dashed] (3)
%                 (1) edge (4);
%         \end{scope}
%     \end{tikzpicture}
%     \hspace*{0.15\textwidth}
%     \begin{tikzpicture}
%         \begin{scope}[every node/.style={circle,thick,draw}]
%             \node (1) [fill=pink] at (0,0) {1};
%             \node (2) at (xyz polar cs:angle=0, radius=2.5) {2};
%             \node (3) at (xyz polar cs:angle=120, radius=2.5) {3};
%             \node (4) at (xyz polar cs:angle=240, radius=2.5) {4};
%         \end{scope}
%         \begin{scope}[every node/.style={colour=black}]
%             \node (1c) [above=1pt of 1, colour=red] {-1};
%             \node (2c) [above=1pt of 2] {0};
%             \node (3c) [above=1pt of 3] {1};
%             \node (4c) [above=1pt of 4] {-1};
%         \end{scope}
%         \begin{scope}[every edge/.style={draw,very thick}]
%             \path
%                 (1) edge [dashed] (2)
%                 (1) edge (3)
%                 (1) edge [dashed] (4);
%         \end{scope}
%     \end{tikzpicture}
%     \caption{Vertex colouring consistency under switching}
% \end{figure}

% \begin{figure}[h]
%     \centering
%     \begin{tikzpicture}
%         \begin{scope}[every node/.style={circle,thick,draw,distance=1pt}, rotate=0]
%             \node (1) [fill=pink, label={[red]10:0}, label={[red]155:-1}, label={[red]270:1}] at (0,0) {1};
%             \node (2) [label=170:0] at (xyz polar cs:angle=0, radius=3) {2};
%             \node (3) [label=270:1] at (xyz polar cs:angle=120, radius=3) {3};
%             \node (4) [label=20:1] at (xyz polar cs:angle=240, radius=3) {4};
%         \end{scope}
%         \begin{scope}[every edge/.style={draw,very thick}]
%             \path
%                 (1) edge (2)
%                 (1) edge [dashed] (3)
%                 (1) edge (4);
%         \end{scope}
%     \end{tikzpicture}
%     \hspace*{0.15\textwidth}
%     \begin{tikzpicture}
%         \begin{scope}[every node/.style={circle,thick,draw,distance=1pt}, rotate=0]
%             \node (1) [fill=pink, label={[red]10:0}, label={[red]155:1}, label={[red]270:-1}] at (0,0) {1};
%             \node (2) [label=170:0] at (xyz polar cs:angle=0, radius=3) {2};
%             \node (3) [label=270:1] at (xyz polar cs:angle=120, radius=3) {3};
%             \node (4) [label=20:1] at (xyz polar cs:angle=240, radius=3) {4};
%         \end{scope}
%         \begin{scope}[every edge/.style={draw,very thick}]
%             \path
%                 (1) edge [dashed] (2)
%                 (1) edge (3)
%                 (1) edge [dashed] (4);
%         \end{scope}
%     \end{tikzpicture}
%     \caption[Edge colouring consistency under switching]{Edge colouring consistency under switching. Note the special behavior of the colour $0$: the sign of edges coloured $0$ is irrelevant.}
% \end{figure}


% \begin{figure}[h]\label{fig:transformation}
%     \centering
%         \begin{tikzpicture}
%             \begin{scope}[every node/.style={circle,thick,draw,fill=lightgray}]
%                 \node (1) at (0,0) {1};
%                 \node (2) at (3,0) {2};
%                 \node (3) at (0,-3) {3};
%             \end{scope}
%             \begin{scope}[every node/.style={circle,thick,draw,fill=cyan}]
%                 \node (4) at (4,-3) {4};
%                 \node (5) at (5,-1) {5};
%                 \node (6) at (1.5,-1.5) {6};
%                 \node (7) at (2,-5) {7};
%             \end{scope}
%             \begin{scope}[every edge/.style={draw,dashed,very thick}]
%                 \path
%                     (1) edge (2)
%                     (3) edge [color=red, ultra thick] (4)
%                     (4) edge [color=blue] (7)
%                     (5) edge [color=red, ultra thick] (2)
%                     (6) edge [color=red, ultra thick] (3);
%             \end{scope}
%             \begin{scope}[every every edge/.style={draw,very thick}]
%                 \path
%                     (1) edge [color=red, ultra thick] (6)
%                     (2) edge [color=red, ultra thick] (4)
%                     (3) edge (1)
%                     (4) edge [color=blue] (5)
%                     (4) edge [color=blue] (6);
%             \end{scope}

%             \draw [ultra thick, ->] (5.5,-2) -- (7,-2);

%             \begin{scope}[every node/.style={circle,thick,draw,fill=lightgray}]
%                 \node (11) at (8,0) {1'};
%                 \node (21) at (11,0) {2'};
%                 \node (31) at (8,-3) {3'};
%             \end{scope}
%             \begin{scope}[every node/.style={circle,thick,draw,fill=cyan}]
%                 \node (41) at (12,-3) {4'};
%                 \node (51) at (13,-1) {5'};
%                 \node (61) at (9.5,-1.5) {6'};
%                 \node (71) at (10,-5) {7'};
%             \end{scope}
%             \begin{scope}[every edge/.style={draw,dashed,very thick}]
%                 \path
%                     (21) edge [color=red, ultra thick] (41)
%                     (11) edge [color=red, ultra thick] (61)
%                     (11) edge (21)
%                     (41) edge [color=blue] (71);
%             \end{scope}
%             \begin{scope}[every edge/.style={draw,very thick}]
%                 \path
%                     (31) edge (11)
%                     (41) edge [color=blue] (51)
%                     (41) edge [color=blue] (61)
%                     (31) edge [color=red, ultra thick] (41)
%                     (51) edge [color=red, ultra thick] (21)
%                     (61) edge [color=red, ultra thick] (31);
%             \end{scope}
%         \end{tikzpicture}
%     \caption[Transformation example]{}
% \end{figure}

% \begin{figu
% \begin{figure}[h]
%     \centering
%     \begin{tikzpicture}
%         \begin{scope}[every node/.style={circle,draw}]
%             \node (1)  at (0,0) {1};
%             \node (2)  at (0,-4) {2};
%             \node (3) at (2,-2) {3};

%             \node (4) at (4,-2) {4};

%             \node (5)  at (6,0) {5};
%             \node (6)  at (6,-4) {6};
%             \node (7) at (8,-2) {7};

%             \node (9) at (10,-2) {8};
%             \node (10)  at (12,0) {9};
%             \node (11)  at (12,-4) {10};
%             \node [below left=3pt of 1, color=red] {9};
%             \node [above left=3pt of 2, color=red] {10};
%             \node [left=3pt of 3, color=red] {8};
%             \node [right=3pt of 4, color=red] {7};
%             \node [below=3pt of 5, color=red] {5};
%             \node [right=3pt of 9, color=red] {3};
%             \node [left=3pt of 7, color=red] {4};
%             \node [above=3pt of 6, color=red] {6};
%             \node [above right=3pt of 11, color=red] {2};
%             \node [below right=3pt of 10, color=red] {1};

%         \end{scope}
%         \node (cheat) at (-1.5,1.2) {};
%         \node (cheat2) at (3.8,-5.2) {};
%         \begin{scope}[every edge/.style={draw,very thick}]
%             \path
%                 (1) edge [double distance=3pt] (2)
%                 (1) edge (3)
%                 (2) edge (3)
%                 (3) edge [dashed] (4)
%                 (4) edge (5)
%                 (4) edge (6)
%                 (5) edge (7)
%                 (6) edge (7)
%                 (7) edge (9)
%                 (9) edge (10)
%                 (9) edge (11)
%                 (10) edge [double distance=3pt] (11);
%         \end{scope}
%     \end{tikzpicture}
%     \caption[Example of an incorrect isomorphism allowed by the weaker definition]{Example of an incorrect isomorphism allowed by the weaker definition. Projecting black vertices onto red vertices doesn't change the sign of cycles and yet it is not a switching isomorphism}
% \end{figure}

% We already know that switching doesn't change the balance of cycles so switching equivalence $\implies$ same cycle balance is trivial. For the other direction we will use the definition of a switching equivalence class.
