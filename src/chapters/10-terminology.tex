\chapter{Terminology}\label{ch:preliminaries}

\section{Graphs}

Here we define the basic terminology used in this thesis.

\begin{definition}
    We write $G$ for a graph and $V(G)$ and $E(G)$ for its vertes set and edge set respectively.
    We assume no graph constraints unless otherwise specified, so loops and duplicit edges are allowed in general.
\end{definition}

\begin{definition}
    We write $e = vw \in E(G)$ to indicate that the edge $e$ of $G$ has endpoints $v$ and $w$.
\end{definition}

\begin{definition}
    A \textit{k-regular} graph is a graph where each vertex has degree $k$.
\end{definition}

\begin{definition}
    A \textit{circle} or a \textit{circuit} is a connected 2-regular subgraph.
\end{definition}

\begin{definition}
    A \textit{cubic} graph is a 3-regular graph.
\end{definition}

\section{Signed graphs}

Signed graphs were introduced by Harary\cite{harary} in 1953 as a model for social networks. 
A signed graph has a value of $+1$ or $-1$ assigned to all edges, so each edge is positive or negative.
They have proved to be a natural generalization of unsigned graphs in many ways and interesting observations may arise
by applying ordinary graph theory to signed graphs.

\begin{definition}
    A \textit{signed graph} is a pair $(G, \Sigma)$; $\Sigma \subseteq E(G)$, where $\Sigma$ is a subset of the edge set of $G$ and contains the negative edges.
\end{definition}

\begin{definition}
    Function $\sigma : E(G) \rightarrow \{+1; -1\}$ denotes the sign of an edge $e$.
\end{definition}

A signed graph can also be defined as a pair $(G, \sigma)$ using the sign function directly,
but I found this definition more natural.

\section{Coloring}

Vertex and edge coloring is a deeply explored topic of graph theory, even in the field of signed graphs.
The research was initiated by Zaslavsky\cite{zaslavsky} in the early 1980s and published in multiple seminary papers.
He defined a vertex n-coloring of a signed graph.

\begin{definition}[Zaslavsky]
    A n-coloring of a signed graph $(G, \Sigma)$ is 
    $\phi : V(E) \rightarrow \{-n, -(n-1), \dots -1, 0, 1, \dots, (n-1, n)\}$
    where for each edge $e = vw \in E(G)$: $\phi (v) \neq \sigma (e) \phi (w)$.
\end{definition}

So each vertex in $G$ is assigned a signed color so that the condition of vertex coloring in unsigned graphs (adapted to signed colors) still stands.

