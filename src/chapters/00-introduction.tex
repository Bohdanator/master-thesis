\chapter*{Introduction}
\addcontentsline{toc}{chapter}{Introduction}
\markboth{Introduction}{Introduction}

Despite the fact that the problem of graph colouring has been studied for a long time and is still being relentlessly studied today, there are still areas to explore. Edge colouring is to this day a novel topic in combination with the concept of signed graphs.

First discovered by the mathematician Frank Harary in 1953 as a model for studying social networks, signed graphs colouring remained an idle topic until 1982 when Thomas Zaslavsky published multiple seminary papers on the topic. Many fundamental results in the study of nowhere-zero flows and the chromatic number of signed graphs have been established only recently and research the problem of edge colouring was started by Richard Behr in 2020. The goal of this thesis is to start a systematic study of 3-edge-colourability of signed cubic graphs.

In the first chapter we will define key concepts in the signed graph theory and offer an overview of the research done so far in this topic. We also describe the relationship to unsigned graphs and how it results in the desirable properties of the color set. In the second chapter we outline how systematic generation of signed graphs that are not 3-edge-colorable is achieved. Details regarding the implementation are also provided. In the third chapter we present our results so far, which include trivial requirements for edge-colorability. Finally, we provide options for future research that can be pursued.
