% Use values that are applicable to you and your thesis.

\newif\ifenglish{}
\englishtrue{} % english language
% \englishfalse{} % slovak language

\newif\ifbachelor{}
% \bachelortrue{} % bachelor's thesis
\bachelorfalse{} % master's thesis

\newif\ifcompsci{}
\compscitrue{} % computer science
% \compscifalse{} % bioinformatics

\newif\ifconsultant{}
% \consultanttrue{} % I have a consultant
\consultantfalse{} % no consultant

\newif\iflogoFMFI{}
% \logoFMFItrue{} % display FMFI logo on the cover and the title page
\logoFMFIfalse{} % no logo

% The year in which you plan to turn the thesis in.
\newcommand{\thesisyear}{2024}

% Tip - use the `\\` to split a long thesis name nicely.
\newcommand{\thesisname}{Edge colouring of signed cubic graphs}

% If you are writing a master's thesis, don't forget to add Bc. here.
\newcommand{\thesisauthor}{Bc. Bohdan~Jóža}

\newcommand{\thesissupervisor}{doc.~RNDr.~Robert~Lukoťka,~PhD.}

\ifconsultant{}
    \newcommand{\thesisconsultant}{doc.~RNDr.~Cyril~Fiktívny,~PhD.}
\fi

\newcommand{\thesisacknowledgments}{
    I would like to thank my supervisor docent Robert Lukoťka for patience and inspiration.
}

\newcommand{\thesisabstractsk}{
    Signované grafy boli vynájdené v roku 1953 vďaka Frankovi Hararymu ako model na štúdium vzťahov ale problém farbenia v tejto oblasti nebol preskúmaný do roku 1982, kedy Thomas Zaslavsky zverejnil prvé výsledky. Veľa základných kameňov teórie grafov však bolo do signovaných grafov premostených len nedávno. V tejto práci prezentujeme algoritmus na generovanie neekvivalentných signovaných grafov, ktoré nie sú hranovo 3-zafarbiteľné spolu s vybranými výsledkami a prvotnou analýzou.
}

\newcommand{\thesisabstracten}{
    Signed graphs were invented by Frank Harary in 1953 as a model for studying social networks but the problem of coloring was not explored until 1982 when Thomas Zaslavsky published his first results. However, much of the graph theory fundamentals was not established until recently. In this thesis we continue the research of 3-edge-colorability of signed graphs. We present an algorithm that generates non-equivalent signed graphs that are not 3-edge-colorable along with its results and preliminary analysis.
}

\newcommand{\thesiskeywordssk}{signovaný graf, kubický graf, hranové farbenie, snark, generovanie neekvivalentných grafov}

\newcommand{\thesiskeywordsen}{signed graph, cubic graph, edge coloring, snark, generating non-equivalent graphs}

\newcommand{\thesischapters}{
    \input chapters/00-introduction.tex
    \input chapters/preliminaries.tex
    \input chapters/generation.tex
    \input chapters/analysis.tex
    \input chapters/conclusion.tex
}

\newcommand{\thesisappendices}{
    \input chapters/appendixA.tex
}

% Thesis type, field, programme and department are set here automatically, but please check that
% the values are indeed correct for you.
% If your supervisor works at FMFI, set the \thesisdepartment to the supervisor's department (this
% one should also be present in the thesis assignment in AIS). Otherwise, keep the Department of
% Computer Science.
\ifenglish{}
    \ifbachelor{}
        \newcommand{\thesistype}{Bachelor's Thesis}
    \else
        \newcommand{\thesistype}{Master's Thesis}
    \fi
    \ifcompsci{}
        \newcommand{\thesisfield}{Computer Science}
        \newcommand{\thesisprogramme}{Computer Science}
    \else
        \newcommand{\thesisfield}{Computer Science and Biology}
        \newcommand{\thesisprogramme}{Bioinformatics}
    \fi
    \newcommand{\thesisdepartment}{Department of Computer Science}
    \newcommand{\thesisfaculty}{Faculty of Mathematics, Physics and Informatics}
    \newcommand{\thesisuniversity}{Comenius University in Bratislava}
\else
    \ifbachelor{}
        \newcommand{\thesistype}{Bakalárska práca}
    \else
        \newcommand{\thesistype}{Diplomová práca}
    \fi
    \ifcompsci{}
        \newcommand{\thesisfield}{Informatika}
        \newcommand{\thesisprogramme}{Informatika}
    \else
        \newcommand{\thesisfield}{Informatika a Biológia}
        \newcommand{\thesisprogramme}{Bioinformatika}
    \fi
    \newcommand{\thesisdepartment}{Katedra informatiky}
    \newcommand{\thesisfaculty}{Fakulta matematiky, fyziky a informatiky}
    \newcommand{\thesisuniversity}{Univerzita Komenského v Bratislave}
\fi
\newcommand{\thesislocation}{Bratislava}

% Helpers

\newif\ifshowframe{}
% \showframetrue{} % show page frames to debug positioning
\showframefalse{} % normal

% EXPERIMENTAL
% When set to true, tries to omit all images, tables, titles, whitespace, etc.
% Useful when measuring the actual amount of text written.
% You might need to run `make -C src clean` beforehand.
\newif\iftextonly{}
% \textonlytrue{} % text-only mode
\textonlyfalse{} % normal
