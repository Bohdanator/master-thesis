% Use values that are applicable to you and your thesis.

\newif\ifenglish{}
\englishtrue{} % english language
% \englishfalse{} % slovak language

\newif\ifbachelor{}
% \bachelortrue{} % bachelor's thesis
\bachelorfalse{} % master's thesis

\newif\ifcompsci{}
\compscitrue{} % computer science
% \compscifalse{} % bioinformatics

\newif\ifconsultant{}
% \consultanttrue{} % I have a consultant
\consultantfalse{} % no consultant

\newif\iflogoFMFI{}
% \logoFMFItrue{} % display FMFI logo on the cover and the title page
\logoFMFIfalse{} % no logo

% The year in which you plan to turn the thesis in.
\newcommand{\thesisyear}{2025}

% Tip - use the `\\` to split a long thesis name nicely.
\newcommand{\thesisname}{Edge colouring of signed cubic graphs}

% If you are writing a master's thesis, don't forget to add Bc. here.
\newcommand{\thesisauthor}{Bc. Bohdan~Jóža}

\newcommand{\thesissupervisor}{doc.~RNDr.~Robert~Lukoťka,~PhD.}

\ifconsultant{}
    \newcommand{\thesisconsultant}{doc.~RNDr.~Cyril~Fiktívny,~PhD.}
\fi

\newcommand{\thesisacknowledgments}{
    I would like to thank my supervisor docent Robert Lukoťka for patience and inspiration.
}

\newcommand{\thesisabstractsk}{
    Signované grafy definoval v roku 1953 Frank Harary ako model na štúdium sociálnych sietí. Problém farbenia signovaných grafov nebol preskúmaný do roku 1982, kedy Thomas Zaslavsky zverejnil prvé výsledky. Prepínanie vrcholov a izomorfizmus rozdeľuje signované grafy do tried ekvivalencie. V tejto práci prezentujeme algoritmus na generovanie prepínavo-izomorfne neekvivalentných signovaných grafov a algoritmus na konverziu problému hranového farbenia na 3SAT. Kombináciou týchto algoritmov vieme generovať malé 3-hranovo-nezafarniteľné signované kubické grafy a formulovať pozorovania o probléme hranového farbenia.
}

\newcommand{\thesisabstracten}{
    Signed graphs were defined by Frank Harary in year 1953 as a model for studying social networks. The problem of colouring, however, was not explored until 1982 when Thomas Zaslavsky published his first results. Vertex switching and isomorphism creates equivalence classes on signed graphs where each graph in a class can be switched and/or projected onto each other graph. In this thesis we present an algorithm for generating non-switching-isomorphic-equivalent signed graphs and an algorithm for edge colouring to 3SAT conversion. Combining these algorithms allows us to generate small non-3-edge-colourable cubic signed graphs and formulate some observations about the problem of edge colouring.
}

\newcommand{\thesiskeywordssk}{signovaný graf, kubický graf, hranové farbenie, snark, prepínanie vrcholov, prepínavo-izomorfne-neekvivalentné grafy, generovanie grafov}

\newcommand{\thesiskeywordsen}{signed graph, cubic graph, edge colouring, snark, vertex switching, non-switching-equivalent graphs, generating graphs}

\newcommand{\thesischapters}{
    \input chapters/00-introduction.tex
    \input chapters/preliminaries.tex
    \input chapters/switching-isomorphism.tex
    \input chapters/generation.tex
    \input chapters/analysis.tex
    \input chapters/conclusion.tex
}

\newcommand{\thesisappendices}{
    \input chapters/appendixA.tex
}

% Thesis type, field, programme and department are set here automatically, but please check that
% the values are indeed correct for you.
% If your supervisor works at FMFI, set the \thesisdepartment to the supervisor's department (this
% one should also be present in the thesis assignment in AIS). Otherwise, keep the Department of
% Computer Science.
\ifenglish{}
    \ifbachelor{}
        \newcommand{\thesistype}{Bachelor's Thesis}
    \else
        \newcommand{\thesistype}{Master's Thesis}
    \fi
    \ifcompsci{}
        \newcommand{\thesisfield}{Computer Science}
        \newcommand{\thesisprogramme}{Computer Science}
    \else
        \newcommand{\thesisfield}{Computer Science and Biology}
        \newcommand{\thesisprogramme}{Bioinformatics}
    \fi
    \newcommand{\thesisdepartment}{Department of Computer Science}
    \newcommand{\thesisfaculty}{Faculty of Mathematics, Physics and Informatics}
    \newcommand{\thesisuniversity}{Comenius University in Bratislava}
\else
    \ifbachelor{}
        \newcommand{\thesistype}{Bakalárska práca}
    \else
        \newcommand{\thesistype}{Diplomová práca}
    \fi
    \ifcompsci{}
        \newcommand{\thesisfield}{Informatika}
        \newcommand{\thesisprogramme}{Informatika}
    \else
        \newcommand{\thesisfield}{Informatika a Biológia}
        \newcommand{\thesisprogramme}{Bioinformatika}
    \fi
    \newcommand{\thesisdepartment}{Katedra informatiky}
    \newcommand{\thesisfaculty}{Fakulta matematiky, fyziky a informatiky}
    \newcommand{\thesisuniversity}{Univerzita Komenského v Bratislave}
\fi
\newcommand{\thesislocation}{Bratislava}

% Helpers

\newif\ifshowframe{}
% \showframetrue{} % show page frames to debug positioning
\showframefalse{} % normal

% EXPERIMENTAL
% When set to true, tries to omit all images, tables, titles, whitespace, etc.
% Useful when measuring the actual amount of text written.
% You might need to run `make -C src clean` beforehand.
\newif\iftextonly{}
% \textonlytrue{} % text-only mode
\textonlyfalse{} % normal
