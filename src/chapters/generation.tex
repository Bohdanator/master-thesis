\chapter{Coloring signed cubic graphs}

To determine the chromatic index of a cubic graph is an NP-complete problem. By extension, determining the chromatic index of a signed cubic graph is also NP-complete, because of the trivial reduction from the signed chromatic index problem to the unsigned chromatic index problem. Instead of designing an algorithm we decided to implement a conversion from the chromatic index problem to 3SAT and using a highly optimized SAT solver in the hope for better effectiveness.

\subsection{Conversion to 3SAT}

For any cubic signed graph $\Gamma$ we will construct a 3SAT formula $F(\Gamma)$ that is satisfiable if and only if the graph is 3-colorable. There will be three literals for each half-edge $ev \in \Sigma _{\Gamma}$, one for each colour from $C_3 = \{-1, 0, 1\}$. Let's call them $x^{-1}_{ev}$, $x^{0}_{ev}$ and $x^{1}_{ev}$. In any evaluation of these literals that satisfy $F$ exactly one of them will be true denoting the colour of the half-edge. This will be guaranteed using three constituent formulas. Let $\Gamma = (G, \sigma)$

$$F_1 (\Gamma) = \bigwedge _{e = vw \in E_{\Gamma}} (x^{-1}_{ev} \vee x^{0}_{ev} \vee x^{1}_{ev}) \wedge (x^{-1}_{ew} \vee x^{0}_{ew} \vee x^{1}_{ew}) $$

The first formula ensures that each half-edge is coloured and is the only one containing clauses of length 3. The next formula will enforce the correctness of the colouring, restricting the colours of half edges that form one complete edge. Illegal signatures for each edge are negated using DeMorgan rules, resulting in a convenient CNF form. No edge can be coloured $0$ on one side and $1$ or $-1$ on the other

$$F_{abs} (e = vw) = \neg (x^{0}_{ev} \land x^{1}_{ew}) \land \neg (x^{1}_{ev} \land x^{0}_{ew}) \land \neg (x^{0}_{ev} \land x^{-1}_{ew}) \land \neg (x^{-1}_{ev} \land x^{0}_{ew})$$
$$F_{abs} (e = vw) = (\neg x^{0}_{ev} \lor \neg x^{1}_{ew}) \land (\neg x^{1}_{ev} \lor \neg x^{0}_{ew}) \land (\neg x^{0}_{ev} \lor \neg x^{-1}_{ew}) \land (\neg x^{-1}_{ev} \lor \neg x^{0}_{ew}) \land $$

and the colours must be the same if the edge is positive and opposite if the edge is negative, which is equivalen to the following.

$$F _{sign} (e = vw) = (\neg x^{-1}_{ev} \lor \neg x^{\sigma (e, w)}_{ew}) \land (\neg x^{1}_{ev} \lor \neg x^{-\sigma (e, w)}_{ew})$$

$$F_2 = \bigwedge _{e = vw \in E} F _{abs} (e) \wedge F _{sign} (e)$$

Lastly we need to ensure that the colouring is proper. Let $N(v) = \{e ~|~ (e,v) \in \Sigma _{\Gamma}\}$ be the set of half-edges incident to $v$.

$$F_3 = \bigwedge _{\substack{v \in V \\ e_1, e_2 \in N(v); ~ e_1 \neq e_2}} (\neg x^{-1}_{e_1v} \lor \neg x^{-1}_{e_2v}) \land (\neg x^{0}_{e_1v} \lor \neg x^{0}_{e_2v}) \land (\neg x^{1}_{e_1v} \lor \neg x^{1}_{e_2v}) $$

Each pair of half-edges with a common vertex must have different colours.

\begin{theorem}
    3SAT formula $F(\Gamma) = F_1 \wedge F_2 \wedge F_3$ constructed in the way described above is satisfiable if and only if $\Gamma$ is 3-colourable.
\end{theorem}

\textit{Proof.} Follows from the construction of $F$ encapsulating all properties of a proper signed 3-colouring. \qed

Note that we don't need to explicitly ensure that for each half-edge exactly one literal is true, in case where multiple literals for different colors are true for the same edge we could choose any color from among them and the coloring would be proper.

\section{Implementation}

We achieved our results using the following implementation. The programming language of choice was C++ over Python due to its speed and a base of tools for graph computation. We implement a simple data structure to represent signed graphs as opposed to nauty and other optimized structures because there is little support for signed graphs "out of the box". Additionally, there is no need to optimize for the graph size. Cubic graphs and unsigned snarks are generated using snarkhunter. Our SAT solver of choice is the winner of the SAT Competition 2020\cite{SAT-Competition-2020-solvers}, kissat. It is a \say{condensed and improved reimplementation of CaDiCaL in C}.
