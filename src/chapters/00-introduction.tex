\chapter*{Introduction}
\addcontentsline{toc}{chapter}{Introduction}
\markboth{Introduction}{Introduction}

The problem of graph colouring has been known for a long time and is still relentlessly being studied today. Even in a problem this wide there are still areas to explore and improve. Edge colouring in combination with the concept of signed graphs remains more or less unexplored.

First discovered by the mathematician Frank Harary in 1953 for studying a question in social psychology, signed graphs remained idle until 1982 when Thomas Zaslavsky published multiple seminary papers on the topic. Many fundamental results in the study of nowhere-zero flows and the chromatic number of signed graphs have been established only recently and research the problem of signed edge colouring was started by Richard Behr in 2020. We expand on Behr's work to automate the process of finding signed edge colorings.

Signed graphs have been proven to be generalizations of simple graphs in many ways. Exponentially many signed graphs can be constructed given a simple underlying graph. We will show that this amount can be drastically reduced due to switching equivalence and isomorphism. Additionally, removing equivalence results in cleaner data for subsequent analysis.

The main result of our work is a database of signed snarks up to eighteen vertices obtained by processing a database of non-isomorphic cubic graphs. For each underlying graph we generate all non-switching-isomorphic signatures, which is an interesting problem in and of itself. This is achieved by transforming the signed graphs into unsigned graphs while preserving switching equivalence and using existing tools based on the automorphism group to filter them for isomorphisms. Then we transform each signature into a 3SAT instance solvable if and only if the signature is 3-edge-colorable. In addition to producing data for bulk analysis it is possible to process specific larger graphs.

In the first chapter we define key concepts in the signed graph theory and describe the current state of research. We also mention the relationship to unsigned graphs and how it affects the colour set and its requirements. In the second chapter we describe the programs that generate non-switching-isomorphic signed graphs and signed snarks. In the third and final chapter we present some results achieved by using these tools and suggest options for future research that can be pursued.
